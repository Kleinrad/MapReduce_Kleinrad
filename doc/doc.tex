\documentclass[12pt, letterpaper]{article}
\usepackage[utf8]{inputenc}
\usepackage[backend=biber]{biblatex}
\usepackage{float}
\usepackage{tikz}
\usepackage{hyperref}
\usepackage[newfloat]{minted}
\usepackage{caption}
\usepackage{dirtree}
\tolerance=1
\emergencystretch=\maxdimen
\hyphenpenalty=10000
\hbadness=10000
\addbibresource{references.bib}

\newenvironment{code}{\captionsetup{type=listing}}{}
\SetupFloatingEnvironment{listing}{name=Source Code}

\graphicspath{{img/}}

\title{MapReduce‐System (37)}
\author{Fabian Kleinrad (07), 5BHIF}
\date{March 2022}

\begin{document}

\begin{titlepage}
\maketitle
\end{titlepage}

\tableofcontents
\newpage

\section{Introduction}

In this project the technology MapReduce is being simulated. Thereby a simple system has been developed to imitate a the functionality of an MapReduce application. All of the functionality in this project is written with C++17 and compiled with the help of the meson build system\footfullcite{meson}. The communication is based on the TCP protocol and realized using the C++ library asio\footfullcite{asio}.\newline
Furthermore to increase performance and usability protocol buffers\footfullcite{protobuf} are utilized. Protocol Buffers enable the serialization of data structures in an efficient manner, which simplifies working with messages sent between parties in the MapReduce system.\newline
To make use of the advantages of using a MapReduce architecture, a simple use-case consisting of counting the number of character occurrences in a plain text document. This kind of application was chosen due to its simplicity, which enables the focus of this project to stay on MapReduce rather than a test application. 

\section{MapReduce}

MapReduce is a programming model developed to decrease computation time of large data sets. It was invented by Google, the reason being the need to compute various kinds of derived data. Examples would be inverted indices or representations of the graph structure of web documents. These applications all have simplicity in common, there are no complex operations needed to accomplish said tasks. Furthermore are these kinds of processes characterized by accepting large amounts of input data and reducing it to fraction of itself. MapReduce presents a solution to parallelization, fault-tolerance, data distribution and load balancing. Thereby it is based on the principle of map and reduce, which are eponymous for the technology.\footfullcite{mapreducePaper}

\subsection{Input Data}

MapReduce processes unstructured or semi-structured data. The system is designed to accept large amounts of data in bulk. In the first step of the MapReduce process, this data is being split into subsets, to allow for data distribution. The way this data is being divided depends on the implementation and the type of input data present. The result of splitting the raw data is a set of key/value pairs.\footfullcite{uniLeibzigMapReduce}

\subsection{Map}

The map function accepts a set of key/value pairs, with the implementation being provided by the user. Result of this phase is once more a set of key/value pairs. These result pairs represent the significant information contained in the input data. These significant data pairs directly influence the result and all unneeded information is being discarded. The name map stems form assigning a quantity attribute which represents the value of the resulting pairs, to an quality attribute.\footcite{uniLeibzigMapReduce}\footcite{mapreducePaper}

\subsection{Reduce}

Sorted key/value pairs get passed to the reduce function, which groups and thereby reduces the set of data points. The logic of this grouping functionality depends on the application MapReduce is used for. For that reason the reduce function is also implemented by the user.\footcite{uniLeibzigMapReduce}

\pagebreak

\subsection{Additional Phases}

\subsubsection{Shuffle}
In most implementations of a MapReduce model an shuffle phase is carried out, between the map and reduce phase. The shuffle phase is used to sort the resulting key/value pairs from the mapping phase. This is done in an effort to group similar keys into clusters which can than be reduced by a single worker.\footcite{uniLeibzigMapReduce}\newline

\subsubsection{Combine}
To reduce the network traffic an additional combine phase can be used after mapping. Thereby the large amount of key/value pairs resulting from the mapping phase get reduced before they get transferred over the network. However, because of this local aggregation of data it is possible to slow down the instead through shuffling optimized process of reducing data.\footfullcite{hadoop}

\section{Classes}

The class structure in this project is oriented in the MapReduce solution Disco\footfullcite{discoDoc}. In contrast to the realization disco provides, only three parties are present in this project. Disco uses a central master to act as an interface between the client and the workers, which in disco are referred to as slaves. Additionally disco has a server role. This server manages a number of workers and acts as an intermediary between the worker and master. However the role of the server has not been realized in this implementation to steer away from high complexity and enable the realization of a reliably working model in the time-span of this project.

\subsection{Master}

The master acts as an interface connecting the clients to workers. Additionally the master acts as the central server accepting and managing all connections. To simplify handling these tasks, the master is supported by the ClientManager and WorkerManager. These run as independent threads and allow for a delimited program structure.

\begin{figure}[h]
	\centering
	\includegraphics[width=0.6\linewidth]{img/Master}
	\caption{Structure of the Master class.}
	\label{fig:classes_Master}
\end{figure}

Figure \ref{fig:classes_Master} depicts the structure of the Master class. The Master contains instances of ClientManager and WorkerManager, which run in separately and take care of the tasks the master needs to handle. This leaves the core master class with an acceptConnection method, which asynchronously accepts asio clients, which are then delegated to either the WorkerManager or ClientManager.    


\subsubsection{ClientManager}

The ClientManager handels MapReduce client connections. Client refers to an human user of the MapReduce system. 

\begin{figure}[h]
	\centering
	\includegraphics[width=0.7\linewidth]{img/ClientManager}
	\caption{Structure of the ClientManager class.}
	\label{fig:classes_ClientManager}
\end{figure}

In order for the master to be able to add connections, the ClientManager provides a $join$ function. To support disconnecting clients, a $leave$ function is available. Both of these functions take an instance of a connection represented by a shared pointer on a ConnectionObject. By this means the ClientManager can keep track of ongoing client-master connections and acts as an communication interface between them.\newline 
Additionally the $registerJob$ and $sendResult$ function are depicted in figure \ref{fig:classes_ClientManager}. These two functions handle job management on the client side of the master. Thereby ongoing jobs are being tracked and upon finishing the client gets sent the resulting data. Lastly the class contains a $generateID$ function, which is used to generate unique ids to identify clients.

\subsubsection{WorkerManager}

The WorkerManager works analogously to the ClientManager, whereas the WorkerManager handles the master-worker communication. The WorkerManager also provides $join$ and $leave$, which are used by the master to delegate connections. 

\begin{figure}[h]
	\centering
	\includegraphics[width=0.8\linewidth]{img/WorkerManager}
	\caption{Structure of the WorkerManager class.}
	\label{fig:classes_WorkerManager}
\end{figure}

Similarly to the ClientManager, the WorkerManager provides a $assignJob$ function. This function acts as the interface between the ClientManager and WorkerManger. The functionality of distributing a job received from a client is thereby realized. To allow for results to be forwarded to the WorkerManager, the functions $mapResult$ and $reduceResult$ exist. These methods are used by the ConnectionSession were communication takes place.\newline
To increase fault-tolerance, WorkerManager provides a $reAssignTask$ function, which is used when a worker fails to properly perform its task. The $generateID$ function works the same as in the ClientManager and is used to uniquely identify workers.  

\subsection{Client}

The Client class represents the link between the user and the MapReduce system. It implements a user interface to enable the user to communicate with the system and place job requests. 

\begin{figure}[h]
	\centering
	\includegraphics[width=0.6\linewidth]{img/Client}
	\caption{Structure of the Client class.}
	\label{fig:classes_Client}
\end{figure}

In order to initiate the connection to the master server, the ClientManager implements a $signOn$ function. Similarly the class also provides a $signOff$ function to terminate the connection. The function $sendJob$, which is depicted in figure \ref{fig:classes_Client} is used to place job request. To enable visualization of job results, the functions $printResultsPlain$ as well as $printResultsHistogram$ are implemented by the ClientManager.

\subsection{Worker}

The Worker is the powerhouse of the MapReduce system. It handles the computation of tasks that are assigned by the master. For this purpose the Worker implements the functions $handleMap$ and $handleReduce$, which can theoretically be replaced by any kind of logic accepting and returning the same data types as MapReduce intends. 

\begin{figure}[h]
 	\centering
 	\includegraphics[width=0.8\linewidth]{img/Worker}
 	\caption{Structure of the Worker class.}
 	\label{fig:classes_Worker}
\end{figure}

The $handleMap$ and $handleReduce$ functions take the type of job, data and job id as parameters. Whereas the $handleMap$ function accepts the raw data in the shape of a string, the $handleReduce$ function is provided with a set of key/value pairs.\newline
To control the connection to the master, the functions $signOn$ and $signOff$ are provided which work analogously to the functions in the ClientManager.

\section{Helper Classes}

\subsection{ConnectionObject}

\subsection{ConnectionSession}

\subsection{Job}

\subsubsection{ActiveJob}

\subsection{MessageQueue}

\subsubsection{QueueItem}

\section{Class-diagram}

\section{Implementation}


\section{Usage}
\label{usage}

\subsection{Command Line Arguments}

\subsubsection{Configuration}


\newpage

\section{Project Structure}


% .bib include & references
\newpage

\printbibliography
\end{document}