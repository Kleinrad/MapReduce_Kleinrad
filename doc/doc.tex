\documentclass[12pt, letterpaper]{article}
\usepackage[utf8]{inputenc}
\usepackage[backend=biber]{biblatex}
\usepackage{float}
\usepackage{tikz}
\usepackage{hyperref}
\usepackage[newfloat]{minted}
\usepackage{caption}
\usepackage{dirtree}
\tolerance=1
\emergencystretch=\maxdimen
\hyphenpenalty=10000
\hbadness=10000
\addbibresource{references.bib}

\newenvironment{code}{\captionsetup{type=listing}}{}
\SetupFloatingEnvironment{listing}{name=Source Code}

\graphicspath{{img/}}

\title{MapReduce‐System (37)}
\author{Fabian Kleinrad (07), 5BHIF}
\date{March 2022}

\begin{document}

\begin{titlepage}
\maketitle
\end{titlepage}

\tableofcontents
\newpage

\section{Introduction}

In this project the technology MapReduce is being simulated. Thereby a simple system has been developed to imitate a the functionality of an MapReduce application. All of the functionality in this project is written with C++17 and compiled with the help of the meson build system\footfullcite{meson}. The communication is based on the TCP protocol and realized using the C++ library asio\footfullcite{asio}. Furthermore to increase performance and usability protocol buffers\footfullcite{protobuf} are utilized.  
\section{MapReduce}

\section{Class-diagram}


\subsection{Classes}


\section{Implementation}


\section{Usage}
\label{usage}

\subsection{Command Line Arguments}

\subsubsection{Configuration}


\newpage

\section{Project Structure}


% .bib include & references
\newpage

\printbibliography
\end{document}